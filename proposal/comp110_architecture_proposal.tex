\documentclass{scrartcl}

\usepackage[hidelinks]{hyperref}
\usepackage[none]{hyphenat}

\title{Essay Proposal}
\subtitle{COMP110 - Computer Architecture Essay}

\author{Emiljano Kurbiba}

\begin{document}

\maketitle

\section*{Topic}

My essay will be on Procedural Level Design in general, comparing models and methods used with the end result of determining whether or not Procedural Level Design is actually a technology that the industry should pay more attention to.
% Uncomment as appropriate:
%   convexity-based collision detection for a 2D game engine.
%   procedural level generation for a 2D platform game.

\section*{Paper 1}
% This is an example! Replace the details with a paper relevant to your chosen topic.
\begin{description}
\item[Title:] Procedural Level Generation Using Occupancy-Regulated Extension
\item[Citation:] \cite{PLDORE}
\item[Abstract:] Existing approaches to procedural level generation
in 2D platformer games are, with some notable exceptions,
procedures designed to do the work of a human game designer.
They offer the usual benefits and disadvantages of AI applied to
a cognitive task: they can work much faster than a human level
designer, and are in some cases able to explore the design space
automatically to find levels with desirable qualities. But they
aren’t able to capture the human creativity that produces the
most interesting level designs, and they are usually very specific
to their particular domain. This paper introduces occupancy regulated
extension (ORE), a general geometry assembly algorithm
that supports human-design-based level authoring at
arbitrary scales.
\item[Web link:] \url{https://games.soe.ucsc.edu/sites/default/files/cig10_043CP2_115.pdf}
\item[Full text link:] \url{https://games.soe.ucsc.edu/sites/default/files/cig10_043CP2_115.pdf}
\item[Comments:] I chose this article because I wanted to explore the benefits and drawbacks of AI through Procedural Level Design. 
\end{description}

\section*{Paper 2}
\begin{description}
\item[Title:] Patterns and Procedural Content Generation
\item[Citation:] \cite{bibtex_key}
\item[Abstract:] Procedural content generation and design patterns could potentially
be combined in several different ways in game design.
This paper discusses how to combine the two, using
automatic platform game level design as an example. The
paper also present work towards a pattern-based level generator
for Super Mario Bros, namely an analysis of the levels
of the original Super Mario Bros game into 23 different patterns.
\item[Web link:] http://julian.togelius.com/Dahlskog2012Patterns.pdf
\item[Full text link:] http://julian.togelius.com/Dahlskog2012Patterns.pdf
\item[Comments:] I chose this paper because I wanted to explore how these design methods are in many ways implemented.
\end{description}

\section*{Paper 3}
\begin{description}
\item[Title:] Procedural Content Generation in Games
\item[Citation:] \cite{bibtex_key}
\item[Abstract:] The terms “procedural” and “generation” imply that we are dealing with computer procedures, or algorithms, that create something. A PCG method can be run by a computer (perhaps with human help), and will output something. A PCG system refers to a system that incorporates a PCG method as one of its part, for example an adaptive game or an AI-assisted game design tool. This book will contain plenty of discussion of algorithms and quite a lot of pseudo code, and most of the exercises that accompany the chapters will involve programming.
\item[Web link:] http://pcgbook.com
\item[Full text link:] http://pcgbook.com
\item[Comments:] Again similar to the first article this book explores AI and since it it one of the hardest features to control with procedural design I wanted to understand how that AI behaves with this specific design method.
\end{description}

\section*{Paper 4}
\begin{description}
\item[Title:] Rhythm-Based Level Generation for 2D Platformers 
\item[Citation:] \cite{RBLG}
\item[Abstract:] We present a rhythm-based method for the automatic generation
of levels for 2D platformers, where the rhythm is that which the
player feels with his hands while playing. Levels are created using
a grammar-based method: first generating rhythms, then
generating geometry based on those rhythms. Generation is
constrained by a set of style parameters tweakable by a human
designer. The approach also minimizes the amount of content that
must be manually authored, instead relying on geometry
components that are included in the level designer’s tileset and a
set of jump types. Our results show that this method produces an
impressive variety of levels, all of which are fully playable. 

\item[Web link:] https://users.soe.ucsc.edu/~ejw/papers/smith-platformer-generation-fdg2009.pdf
\item[Full text link:] https://users.soe.ucsc.edu/~ejw/papers/smith-platformer-generation-fdg2009.pdf
\item[Comments:] This article will help me understand the many ways in which human feedback can boost this design method making end game playability superior to others of similar methods.
\end{description}

\section*{Paper 5}
\begin{description}
\item[Title:] Procedural Level Design for Platform Games 
\item[Citation:] \cite{PLDPG}
\item[Abstract:] Although other genres have used procedural level
generation to extend gameplay and replayability, platformer
games have not yet seen successful level generation. This
paper proposes a new four-layer hierarchy to represent
platform game levels, with a focus on representing
repetition, rhythm, and connectivity. It also proposes a way
to use this model to procedurally generate new levels. 

\item[Web link:] http://aaaipress.org/Papers/AIIDE/2006/AIIDE06-022.pdf
\item[Full text link:] http://aaaipress.org/Papers/AIIDE/2006/AIIDE06-022.pdf
\item[Comments:] This article helps me explore a new design method/model I personally wanted to just explore the different kind of models and how they will help me better understand PLD.
\end{description}

\begin{thebibliography}{9}

\bibitem{PLDORE} 
	Peter Mawhorter, Michael Mateas (2010) Procedural Level Generation Using Occupancy-Regulated Extension. United States: IEEE Conference on Computational Intelligence and Games.

\bibitem{PPCG} 
Steve Dahlskog, Julian Togelius (2012) Patterns and Procedural Content Generation. United States: Malmö University, IT University of Copenhagen.

\bibitem{PCGIG} 
Shaker, Noor and Togelius, Julian and Nelson, Mark J (2015) Procedural Content Generation in Games: A Textbook and an Overview of Current Research. Springer.

\bibitem{RBLG} 
Gillian Smith, Mike Treanor, Jim Whitehead, Michael Mateas (2009) Rhythm-Based Level Generation for 2D Platformers. Expressive Intelligence Studio, University of California, Santa Cruz.

\bibitem{PLDPG} 
Kate Compton and Michael Mateas (2006) Procedural Level Design for Platform Games: Literature, Communication \& Culture and College of Computing, Georgia Institute of Technology.

\end{thebibliography}

\end{document}
